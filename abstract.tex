\newpage
{\Huge \bf Abstract}
\vspace{24pt} 


Data intensive computing frameworks have become ubiquitous due to their ability
to crunch petascale data while scaling to thousands of machines. These
frameworks require high throughput to transfer huge amounts of data during
different stages. In fact, transfer of this data dictates job completion time by
more than 50\%. Unfortunately, the troika of data center topologies, protocols,
and control mechanisms has been unable to satisfy these high throughput needs.
In addition, lack of proper network behaviour understanding has forced all of
these frameworks to use standard TCP for data transfer which is sub-optimal in
the data center due to the radically different environment. Recently proposed
extensions of TCP address some of these shortcomings but are unable to provide a
general solution.

To remedy this, we present Mission Control, a system that uses application
requirements in conjunction with system-wide state to choose the best possible
mechanism and protocol for data transfer. Specifically, to optimize data
transfer it, (1) Uses different policies to distribute available network
resources among the running tasks and (2) Chooses the transport protocol based
on user-constraints or network characteristics. We have implemented Mission
Control for CIEL and the framework currently supports a wide range of resource
allocation policies such as fair, FIFO, priority, and performance-centric.
Additionally, the list of supported transport protocols includes data center
TCP, multi-path TCP, and TCPcrypt. Moreover, the framework can distinguish
between shuffle and broadcast transfer to appropriately allocate resources.

Our evaluation shows that data transfer indeed accounts for a large chunk of the
job completion time and can potentially become a performance bottle-neck. In
addition, multi-path TCP enables the framework to leverage multiple paths in
the network. Moreover, data center TCP improves job completion time by 20\% on
average but is unable to match TCP's per flow fairness. Furthermore, TCPcrypt
can transparently encrypt data with an acceptable overhead. Most importantly,
the evaluation corroborates our thesis that there is no single solution which is
applicable across the board and to counter this, it imperative to have diversity
at the transport layer.



\newpage
\vspace*{\fill}
