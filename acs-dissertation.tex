%% 
%% ACS project dissertation template. 
%% 
%% Currently designed for printing two-sided, but if you prefer to 
%% print single-sided just remove ",twoside,openright" from the 
%% \documentclass[] line below. 
%%
%%
%%   SMH, May 2010. 


\documentclass[a4paper,12pt,twoside,openright]{report}


%%
%% EDIT THE BELOW TO CUSTOMIZE
%%

\def\authorname{Zubair Nabi\xspace}
\def\authorcollege{Robinson College\xspace}
\def\authoremail{Zubair.Nabi@cl.cam.ac.uk}
\def\dissertationtitle{Mission Control: Enabling Efficient and Heterogeneous Data Transfers
in Data Intensive Computing}
\def\wordcount{14,235}


\usepackage{epsfig,graphicx,parskip,setspace,tabularx,xspace,epstopdf} 

%% START OF DOCUMENT
\begin{document}


%% FRONTMATTER (TITLE PAGE, DECLARATION, ABSTRACT, ETC) 
\pagestyle{empty}
\singlespacing
\input{titlepage}
\onehalfspacing
\input{declaration}
\singlespacing
\newpage
{\Huge \bf Abstract}
\vspace{24pt} 


Data intensive computing frameworks have become ubiquitous due to their ability
to crunch petascale data while scaling to thousands of machines. These
frameworks require high throughput to transfer huge amounts of data during
different stages. In fact, transfer of this data dictates job completion time by
more than 50\%. Unfortunately, the troika of data center topologies, protocols,
and control mechanisms has been unable to satisfy these high throughput needs.
In addition, lack of proper network behaviour understanding has forced all of
these frameworks to use standard TCP for data transfer which is sub-optimal in
the data center due to the radically different environment. Recently proposed
extensions of TCP address some of these shortcomings but are unable to provide a
general solution.

To remedy this, we present Mission Control, a system that uses application
requirements in conjunction with system-wide state to choose the best possible
mechanism and protocol for data transfer. Specifically, to optimize data
transfer it, (1) Uses different policies to distribute available network
resources among the running tasks and (2) Chooses the transport protocol based
on user-constraints or network characteristics. We have implemented Mission
Control for CIEL and the framework currently supports a wide range of resource
allocation policies such as fair, FIFO, priority, and performance-centric.
Additionally, the list of supported transport protocols includes data center
TCP, multi-path TCP, and TCPcrypt. Moreover, the framework can distinguish
between shuffle and broadcast transfers to appropriately allocate resources.

Our evaluation shows that data transfer indeed accounts for a large chunk of the
job completion time and can potentially become a performance bottle-neck. In
addition, multi-path TCP enables the framework to leverage multiple paths in the
network. Moreover, data center TCP improves job completion time by 20\% on
average but is unable to match TCP's per flow fairness. Furthermore, TCPcrypt
can transparently encrypt data with an acceptable overhead. Most importantly,
the evaluation corroborates our thesis that there is no single solution which is
applicable across the board and to counter this, it is imperative to have
diversity at the transport layer.



\newpage
\vspace*{\fill}


\pagenumbering{roman}
\setcounter{page}{0}
\pagestyle{plain}
\tableofcontents
\listoffigures
\listoftables

\onehalfspacing

%% START OF MAIN TEXT 

\chapter{Introduction}
\pagenumbering{arabic} 
\setcounter{page}{1} 

Over the course of the last decade, proliferation in web applications (Facebook,
Twitter etc.), and scientific computing (sensor networks, high energy physics,
etc.) has resulted in an unparalleled surge in data on the exabyte scale.
Naturally, the need to store and analyze this data has engendered an ecosystem
of distributed file systems~\cite{Ghemawat:2003:GFS}, structured and
unstructured data stores~\cite{Chang:2006:BDS,DeCandia:2007:DAH}, and
data-intensive computing
frameworks~\cite{Dean:2004:MSD,Isard:2007:DDD,Murray:2011:CUE}. These
``shared-nothing'' frameworks are mostly run in massive data centers with tens
of thousands of servers and network elements. These data centers are
geo-located, for diversity and redundancy, and run a wide-variety of loads,
including user-facing applications, such as email, custom enterprise
applications, such as authentication services, and large-scale batch-processing
applications, such as web indexing~\cite{Benson:2010:NTC}. The goal of each data
center is to provide multi-tenant environments for energy, power, space, and
cost-efficient computation, and storage at scale~\cite{Katz:2009:TTB}.

Typically, data centers are built using commodity, off-the-shelf servers and
switches to find a sweet spot between performance and cost and leverage
economies of scale~\cite{Barroso:2003:WSP}. Servers in this model, store or
process a \emph{chunk} of data, with redundancy ensured via replication without
any cluster/data center wide memory or storage area network (SAN). Due to their
commodity nature, servers and network elements are failure-prone and a failure
in say, a \emph{core switch} can bring down the entire network. Therefore,
storage and processing frameworks use a range of methods to deal with this,
including replication, re-execution, speculative execution. On the network side,
the infrastructure predominantly consists of TCP/IP enabled wired links
connecting cheap switches with shared memory, although
wireless~\cite{Halperin:2011:ADC} and
optical~\cite{Wang:2010:CPO,Farrington:2010:HHE} links have recently been
explored. Most data centers are provisioned with 1Gbps and 10Gbps switches with
L2 Ethernet switched fabric interconnects as opposed to fabrics from the high
performance computing community such as InfiniBand. Likewise, due to the
prohibitive cost of high-end switches and routers, bandwidth is a constrained
resource in the data center. To ensure optimum utilization, subject to
organizational goals, oversubscription factors\footnote{Oversubscription factor
can simply be defined as the amount of bandwidth seen by a particular
application as opposed to the total available
bandwidth~\cite{Al-Fares:2008:SCD}.} in real data centers vary significantly:
Some are as low as 2:1 while some can get as high as
147:1~\cite{Benson:2010:NTC}. In the cluster hierarchy, bandwidth increases from
the edge of the network towards the core (off-rack oversubscription). As a
result of off-rack oversubscription, applications are designed to keep
cross-rack communication to a minimal~\cite{Dean:2004:MSD}.

Concurrently, the emergence of cloud computing has resulted in multi-tenant data
centers in which user applications have wide-varying requirements and
communication patterns. Applications range from low-latency query jobs to
bandwidth hungry batch processing jobs~\cite{Alizadeh:2010:DCT}. In addition,
applications exhibit one-to-many, many-to-one, many-to-many communication
patterns. On the data center fabric level, to ensure better fault-tolerance,
scalability, and end-to-end bandwidth, several new
topologies~\cite{Al-Fares:2008:SCD,Guo:2008:DSF,Guo:2009:BHP,Greenberg:2009:VSF}
have been proposed to replace the status quo: a hierarchical 2/3-level tree
which connects end-hosts through \emph{top-of-rack} (or \emph{aggregation}), and
\emph{core} switches. In the same vein,
L2~\cite{Mudigonda:2010:SCD,Vattikonda:2012:PTD} and
L3/L4~\cite{Alizadeh:2010:DCT,Vasudevan:2009:SEF,Raiciu:2010:DCN,Wilson:2011:BNL,Wu:2010:IIC}
protocols, topologies and design
frameworks~\cite{Singla:2011:JND,Al-Fares:2008:SCD,Guo:2008:DSF,Guo:2009:BHP,Greenberg:2009:VSF,Mudigonda:2011:TFC,Chen:2010:GAA},
and control and virtualization
planes~\cite{NiranjanMysore:2009:PSF,Mudigonda:2011:NSM,Guo:2010:SDC,Ballani:2011:TPD,Shieh:2011:SDC,Rodrigues:2011:GSB,Al-Fares:2010:HDF}
for data center communication and resource allocation have also experienced
innovation for varying goals picked from a set of high bandwidth, fairness,
fault-tolerance, and scalability.

Above all, the ossification of the Internet protocol stack at the transport and
network layer~\cite{Akhshabi:2011:ELP} has also been mimicked by data center
networks. TCP/IP has become the de-facto ``narrow-waist" of the data center
network. On the one hand, this has standardized direct communication to/from and
within the data center without any middleware, but on the other, it has also
spawned a new set of problems unique to the data center environment. In the data
center, the role of IP at the network layer is minimal, as it is only employed
for naming and addressing. The only obstacle is address assignment which needs
to be done manually to encode location/topology information within the address.
Although, manual assignment is tedious and error-prone, it can be automated by
leveraging the well-defined structure of data centers~\cite{Chen:2010:GAA}. In
contrast, TCP faces a number of pathological challenges due to a significantly
different bandwidth-delay product, round-trip time (RTT), and retransmission
timeout (RTO)~\cite{Chen:2009:UTI} than a wide area network (WAN). For example,
due to the low RTT, the congestion window for each flow is very small. As a
result, flow recovery through TCP fast retransmit is impossible, leading to poor
net throughput~\cite{Kandula:2009:NDC}. To exacerbate problems, data-intensive
computing frameworks (DICF) are bandwidth hungry and require all data to be
\emph{materialized} before the next stage can commence (For example,
\emph{shuffle} between Map and Reduce stages). In fact, data transfer between
stages can account for more than 50\% of the job completion
time~\cite{Chowdhury:2011:MDT}. Similar to compute, memory, and disk I/O
``stragglers''~\cite{Zaharia:2008:IMP}, an imbalance in network efficiency can
lead to the job completion time being dictated by the task with the longest
shuffle time.

It is important to highlight that some of these problems stated above, stem in
part due to the difference in traffic characteristics and scale between data
centers and other networks. In addition, unlike wide area network applications,
there is a tight coupling between an application's usage of network, computing,
and storage resources. In production data centers, due to wide-varying mix of
applications, congestion in the network can last from 10s to 100s of
seconds~\cite{Kandula:2009:NDC}. Moreover, during high utilization periods, the
task failure rate of batch-processing systems increases, especially during the
\emph{reduce} phase of MapReduce-style applications, by a median value of 1.1x,
due to read failures over the network. In the worst case, this can halt the
entire job if subsequent tasks are dependent on the failed ones. Furthermore,
link utilization is highest in the core of the network, but most losses occur
towards the edge~\cite{Benson:2010:NTC}. Additionally, in some cases
over-provisioning the network is overkill and good performance can be achieved
by adding a few extra links at hotspot points~\cite{Kandula:2009:FTD}. Finally,
in commodity switches the buffer pool is shared by all interfaces. As a result
if long flows hog the memory, there will be queue build up in the short flows.
In reaction, TCP reduces the window size by half based on the presence of
congestion regardless of its extent. This leads to a large mismatch between the
input rate and the capacity of the link leading to buffer underflows and loss of
throughput~\cite{Alizadeh:2010:DCT}. Overall, traffic engineering in the data
center is a complex task due to the wide variability in flow duration, flow
size, flow arrival time, and server participation~\cite{Kandula:2009:NDC}, and
standard TCP without any global knowledge is suboptimal~\cite{Benson:2010:NTC}
[Look up citation].

Another major shortcoming of TCP which is enunciated in the data center is
\emph{Incast}~\cite{Chen:2009:UTI,Vasudevan:2009:SEF,Wu:2010:IIC,Alizadeh:2010:DCT}.
As a result of which, the overall application goodput significantly goes down.
Typically, to handle incast, application level remedies are added which include
varying the size of the packets so that a large number of packets can fit the
switch memory, and adding application-level jitter to offset the packet
inter-arrival time. Both of these solutions are suboptimal, as they either,
increase the number of packets, or add unnecessary
delay~\cite{Alizadeh:2010:DCT}.

TCP was designed for a wide area network (WAN).

Therefore, in recent years, a number of reverse-engineered versions of TCP have
been proposed. Unfortunately, most solutions are tied either to the underlying
topology or the traffic mix of the data center. There is no one-size-fits-all
solution. Therefore, there needs to be a decoupling between data transfer policy
and mechanism. The same strategy when applied at the network layer improves both
application-level and network-level performance~\cite{Abu-Libdeh:2010:SRF}. In
addition to improving performance, transport layer characteristics an also be
chosen to improve energy efficiency~\cite{Heller:2010:ESE} or
security~\cite{bittau:the}. Adding power constraints can greatly reduce the
consumption footprint of today's data centers where the network consumes 10-20\%
of the total power~\cite{Greenberg:2008:CCR}.
Likewise, application traffic can transparently and
efficiently~\cite{bittau:the} be encrypted to ensure security in a multi-tenant
environment.


Existing solutions which ensure fairness and high bandwidth, work at the flow or
packet level which does not capture the needs of DICF where the unit of
communication is a \emph{transfer}.
Put differently, stages between DICF are dependent on bulk \emph{transfers} of
data. A recent system, Orchestra~\cite{Chowdhury:2011:MDT}, works at the
application layer to enable intra and inter-transfer control of two
communication patterns, which lie at the heart of DICF: \emph{shuffles}, and
\emph{broadcasts}. For the former it uses weighted flow assignment (WFA) enabled
by an appropriate number of multiple TCP connections and TCP's AIMD fair sharing
while for the latter it leverages a BitTorrent-like algorithm. While the results
from Orchestra are promising, it suffers from a number of shortcomings. Most
notably, using single-flow TCP for WFA is suboptimal due to TCP's inefficiency
in (a) moving away from congested links, and (b) exploiting denseness of
interconnection in advanced
topologies~\cite{Al-Fares:2008:SCD,Guo:2008:DSF,Guo:2009:BHP,Greenberg:2009:VSF}
as noted by Raiciu et al.~\cite{Raiciu:2011:IDP}. In addition, some applications
might not require TCP and could possibly benefit from other transport
protocols~\cite{bittau:the,Alizadeh:2010:DCT,Vasudevan:2009:SEF,Raiciu:2010:DCN,Wilson:2011:BNL}.
Finally, frameworks like Dryad support TCP pipes, and shared memory FIFO for
communication in addition to transfer of files over the network, while Orchestra
only considers the last of these.

This is the introduction where you should introduce your work.  In
general the thing to aim for here is to describe a little bit of the
context for your work --- why did you do it (motivation), what was the
hoped-for outcome (aims) --- as well as trying to give a brief
overview of what you actually did.

It's often useful to bring forward some ``highlights'' into 
this chapter (e.g.\ some particularly compelling results, or 
a particularly interesting finding). 

It's also traditional to give an outline of the rest of the
document, although without care this can appear formulaic 
and tedious. Your call. 


\chapter{Background} 

A more extensive coverage of what's required to understand your 
work. In general you should assume the reader has a good undergraduate 
degree in computer science, but is not necessarily an expert in 
the particular area you've been working on. Hence this chapter 
may need to summarize some ``text book'' material. 

This is not something you'd normally require in an academic paper, 
and it may not be appropriate for your particular circumstances. 
Indeed, in some cases it's possible to cover all of the ``background'' 
material either in the introduction or at appropriate places in 
the rest of the dissertation. 


\chapter{Related Work} 

Because of the shared nature of the new topologies the full bisection bandwidth
is not available[cite Hedara/Camdoop]? They also increase the overall cost and
wiring complexity.

CamCube proposes a 3D torus topology in which each server is directly connected
to a small set of other servers inspired by structured
overlays~\cite{Abu-Libdeh:2010:SRF}. This scheme places each server into a 3D
wrapped coordinate space and its position (\emph{x,y,z}) in the space represents
its address. As a result, the physical and virtual topology of the network is
the same and applications can infer their location through an API which exposes
coordinate space information and allows one-hop Ethernet packets. Thus,
applications can implement their own routing strategies. Mission Control can be
used on CamCube to enable applications to choose both their routing strategies
and their transport mechanism.

\section{Reducing Network Traffic}
Another point in the design space is the reduce the amount of traffic that is
transferred across the network by MapReduce-like
systems~\cite{Costa:2012:CEI,Yu:2009:DAD}. This can be done, either at different
levels in the cluster topology~\cite{Yu:2009:DAD} or made a function of the
network itself as in Camdoop~\cite{Costa:2012:CEI}. On the downside, performing
rack-level aggregation can saturate the ingress link to the aggregating
server~\cite{Yu:2009:DAD}. This \emph{distributed combiner} approach is most
effective for MapReduce functions which are commutative and
associative~\cite{Dean:2004:MSD}. For other function types, Camdoop distributes
the implicit sort and merge of MapReduce across all servers. At the same time, a
custom transport layer performs global scheduling of packets. This solution is
only applicable to a CamCube topology but shows that custom transport protocols
can greatly enhance performance.

\section{Wireless and Optical Networks}
Another line of work has argued that providing full-bisection links throughout
the network is overkill as only a few hotspots exist in actual data
centers~\cite{{Halperin:2011:ADC,Kandula:2009:FTD}}. The high cost and technical
challenges of newer topologies can be avoided by provisining the network for the
average case and avoiding hotspots by adding additional links on the fly.
The key challenge is to choose the placement and duration of \emph{flyways}.
Flyways are constructed using Multi-Gigabit wireless links, controlled by a
centralized scheduler that decides the temporal and spatial placement of
wireless links based on observed traffic. In cases where the traffic has high
churn, the controller would fail to respond quickly.

\section{Energy Efficiency and Security}
ElasticTree is a system that dynamically alters the power consumption of
networking elements inside a data center based on the current
traffic~\cite{Heller:2010:ESE}. It consists of an optimizer, routing control,
and power control. Based on the topology, traffic matrix, other inputs, the
optimizer computes an optimum power network subnet which is in turn fed to the
power control and the routing control. The former toggles the power state of
network elements, and the latter chooses paths for each flow. The optimizer
provides a number of strategies which make a trade-off between scalability, and
optimality. ElasticTree can be used in conjunction with Mission Control to
choose an underlying transport protocol that behaves well with it. For example,
the performance of TCP is adversely affected if there is packet reordering due
to the splitting for a single flow across multiple links by ElasticTree.

TCPcrypt is an enhancement of TCP that uses 


This chapter covers relevant (and typically, recent) research 
which you build upon (or improve upon). There are two complementary 
goals for this chapter: 
\begin{enumerate} 
  \item to show that you know and understand the state of the art; and 
  \item to put your work in context
\end{enumerate} 

Ideally you can tackle both together by providing a critique of
related work, and describing what is insufficient (and how you do
better!)

The related work chapter should usually come either near the front or
near the back of the dissertation. The advantage of the former is that
you get to build the argument for why your work is important before
presenting your solution(s) in later chapters; the advantage of the
latter is that don't have to forward reference to your solution too
much. The correct choice will depend on what you're writing up, and
your own personal preference.



\chapter{Design and Implementation} 

This chapter may be called something else\ldots but in general 
the idea is that you have one (or a few) ``meat'' chapters which
describe the work you did in technical detail. 


\chapter{Evaluation} 

For any practical projects, you should almost certainly have
some kind of evaluation, and it's often useful to separate 
this out into its own chapter. 


\chapter{Summary and Conclusions} 

As you might imagine: summarizes the dissertation, and draws 
any conclusions. Depending on the length of your work, and 
how well you write, you may not need a summary here. 

You will generally want to draw some conclusions, and point
to potential future work. 

Possibly merge with Mesos~\cite{Hindman:2011:MPF} and
Akaros~\cite{Rhoden:2011:IPE} or Quincy~\cite{Isard:2009:QFS}.



\appendix
\singlespacing

\bibliographystyle{unsrt} 
\bibliography{dissertation} 

\end{document}
