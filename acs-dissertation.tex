%% 
%% ACS project dissertation template. 
%% 
%% Currently designed for printing two-sided, but if you prefer to 
%% print single-sided just remove ",twoside,openright" from the 
%% \documentclass[] line below. 
%%
%%
%%   SMH, May 2010. 


\documentclass[a4paper,12pt,twoside,openright]{report}


%%
%% EDIT THE BELOW TO CUSTOMIZE
%%

\def\authorname{Zubair Nabi\xspace}
\def\authorcollege{Robinson College\xspace}
\def\authoremail{Zubair.Nabi@cl.cam.ac.uk}
\def\dissertationtitle{Mission Control: Enabling Efficient and Heterogeneous Data Transfers
in Data Intensive Computing}
\def\wordcount{14,235}


\usepackage{epsfig,graphicx,parskip,setspace,tabularx,xspace,epstopdf} 

%% START OF DOCUMENT
\begin{document}


%% FRONTMATTER (TITLE PAGE, DECLARATION, ABSTRACT, ETC) 
\pagestyle{empty}
\singlespacing
\input{titlepage}
\onehalfspacing
\input{declaration}
\singlespacing
\newpage
{\Huge \bf Abstract}
\vspace{24pt} 


Data intensive computing frameworks have become ubiquitous due to their ability
to crunch petascale data while scaling to thousands of machines. These
frameworks require high throughput to transfer huge amounts of data during
different stages. In fact, transfer of this data dictates job completion time by
more than 50\%. Unfortunately, the troika of data center topologies, protocols,
and control mechanisms has been unable to satisfy these high throughput needs.
In addition, lack of proper network behaviour understanding has forced all of
these frameworks to use standard TCP for data transfer which is sub-optimal in
the data center due to the radically different environment. Recently proposed
extensions of TCP address some of these shortcomings but are unable to provide a
general solution.

To remedy this, we present Mission Control, a system that uses application
requirements in conjunction with system-wide state to choose the best possible
mechanism and protocol for data transfer. Specifically, to optimize data
transfer it, (1) Uses different policies to distribute available network
resources among the running tasks and (2) Chooses the transport protocol based
on user-constraints or network characteristics. We have implemented Mission
Control for CIEL and the framework currently supports a wide range of resource
allocation policies such as fair, FIFO, priority, and performance-centric.
Additionally, the list of supported transport protocols includes data center
TCP, multi-path TCP, and TCPcrypt. Moreover, the framework can distinguish
between shuffle and broadcast transfers to appropriately allocate resources.

Our evaluation shows that data transfer indeed accounts for a large chunk of the
job completion time and can potentially become a performance bottle-neck. In
addition, multi-path TCP enables the framework to leverage multiple paths in the
network. Moreover, data center TCP improves job completion time by 20\% on
average but is unable to match TCP's per flow fairness. Furthermore, TCPcrypt
can transparently encrypt data with an acceptable overhead. Most importantly,
the evaluation corroborates our thesis that there is no single solution which is
applicable across the board and to counter this, it is imperative to have
diversity at the transport layer.



\newpage
\vspace*{\fill}


\pagenumbering{roman}
\setcounter{page}{0}
\pagestyle{plain}
\tableofcontents
\listoffigures
\listoftables

\onehalfspacing

%% START OF MAIN TEXT 

\chapter{Introduction}
\pagenumbering{arabic} 
\setcounter{page}{1} 

Over the course of the last decade, proliferation in web applications (Facebook,
Twitter etc.), and scientific computing (sensor networks, high energy physics,
etc.) has resulted in an unparalleled surge in data on the exabyte scale.
Naturally, the need to store and analyze this data has engendered an ecosystem
of distributed file systems~\cite{Ghemawat:2003:GFS}, structured and
unstructured data stores~\cite{Chang:2006:BDS,DeCandia:2007:DAH}, and
data-intensive computing
frameworks~\cite{Dean:2004:MSD,Isard:2007:DDD,Murray:2011:CUE}. These
``shared-nothing'' frameworks are mostly run in massive data centers with tens
of thousands of servers and network elements. 

Typically, data centers are built using commodity off-the-shelf
 servers and switches to find a sweet spot between performance and
cost~\cite{Barroso:2003:WSP}. Servers in this model store or process a
\emph{chunk} of data with redundancy ensured via replication without any
cluster/data center wide memory or SAN. Due to their commodity nature, servers
and network elements are failure-prone and a failure in say, a \emph{core
switch} can bring down the entire network. Therefore, storage and processing
frameworks use a range of methods to deal with this, such as replication,
re-execution, speculative execution, etc. The network infrastructure
predominantly consists of wired links cheap switches with shared memory,
although wireless~\cite{Halperin:2011:ADC} and
optical~\cite{Wang:2010:CPO,Farrington:2010:HHE} links have recently been
explored. Most data centers are provisioned with 1Gbps and 10Gbps switches with
L2 Ethernet switched fabric interconnects as opposed to fabrics from the high
performance computing community such as InfiniBand. Likewise, due to the
prohibitive cost of high-end switches and routers, bandwidth is a constrained
resource in the data center. In the cluster hierarchy, bandwidth increases from
the edge of the network towards the core (off-rack oversubscription).

The emergence of cloud computing has resulted in multi-tenant data centers in
which user applications have wide-varying requirements and communication
patterns. Applications range from low-latency query jobs to bandwidth hungry
batch processing jobs~\cite{Alizadeh:2010:DCT}. In addition, applications
exhibit one-to-many, many-to-one, many-to-many communication patterns. On the
data center fabric level, to ensure better fault-tolerance, scalability, and
end-to-end bandwidth, several new
topologies~\cite{Al-Fares:2008:SCD,Guo:2008:DSF,Guo:2009:BHP,Greenberg:2009:VSF}
have been proposed to replace the status quo: a hierarchical 2/3-level tree
which connects end-hosts through \emph{top-of-rack} (or \emph{aggregation}), and
\emph{core} switches. In the same vein,
L2~\cite{Mudigonda:2010:SCD,Vattikonda:2012:PTD} and
L3~\cite{Alizadeh:2010:DCT,Vasudevan:2009:SEF,Raiciu:2010:DCN,Wilson:2011:BNL,Wu:2010:IIC}
protocols, topologies and design
frameworks~\cite{Singla:2011:JND,Al-Fares:2008:SCD,Guo:2008:DSF,Guo:2009:BHP,Greenberg:2009:VSF,Mudigonda:2011:TFC,Chen:2010:GAA},
and control and virtualization
planes~\cite{NiranjanMysore:2009:PSF,Mudigonda:2011:NSM,Guo:2010:SDC,Ballani:2011:TPD,Shieh:2011:SDC,Rodrigues:2011:GSB,Al-Fares:2010:HDF}
for data center communication and resource allocation have also experienced
innovation for varying goals picked from a set of high bandwidth, fairness,
fault-tolerance, and scalability.


This is the introduction where you should introduce your work.  In
general the thing to aim for here is to describe a little bit of the
context for your work --- why did you do it (motivation), what was the
hoped-for outcome (aims) --- as well as trying to give a brief
overview of what you actually did.

It's often useful to bring forward some ``highlights'' into 
this chapter (e.g.\ some particularly compelling results, or 
a particularly interesting finding). 

It's also traditional to give an outline of the rest of the
document, although without care this can appear formulaic 
and tedious. Your call. 


\chapter{Background} 

A more extensive coverage of what's required to understand your 
work. In general you should assume the reader has a good undergraduate 
degree in computer science, but is not necessarily an expert in 
the particular area you've been working on. Hence this chapter 
may need to summarize some ``text book'' material. 

This is not something you'd normally require in an academic paper, 
and it may not be appropriate for your particular circumstances. 
Indeed, in some cases it's possible to cover all of the ``background'' 
material either in the introduction or at appropriate places in 
the rest of the dissertation. 


\chapter{Related Work} 

This chapter covers relevant (and typically, recent) research 
which you build upon (or improve upon). There are two complementary 
goals for this chapter: 
\begin{enumerate} 
  \item to show that you know and understand the state of the art; and 
  \item to put your work in context
\end{enumerate} 

Ideally you can tackle both together by providing a critique of
related work, and describing what is insufficient (and how you do
better!)

The related work chapter should usually come either near the front or
near the back of the dissertation. The advantage of the former is that
you get to build the argument for why your work is important before
presenting your solution(s) in later chapters; the advantage of the
latter is that don't have to forward reference to your solution too
much. The correct choice will depend on what you're writing up, and
your own personal preference.



\chapter{Design and Implementation} 

This chapter may be called something else\ldots but in general 
the idea is that you have one (or a few) ``meat'' chapters which
describe the work you did in technical detail. 


\chapter{Evaluation} 

For any practical projects, you should almost certainly have
some kind of evaluation, and it's often useful to separate 
this out into its own chapter. 


\chapter{Summary and Conclusions} 

As you might imagine: summarizes the dissertation, and draws 
any conclusions. Depending on the length of your work, and 
how well you write, you may not need a summary here. 

You will generally want to draw some conclusions, and point
to potential future work. 

Possibly merge with Mesos~\cite{Hindman:2011:MPF} and
Akaros~\cite{Rhoden:2011:IPE}.



\appendix
\singlespacing

\bibliographystyle{unsrt} 
\bibliography{dissertation} 

\end{document}
